\section{Power and Performance Analysis}

本稿では,GPUとCPUにおける電圧と動作周波数を変化させた場合における性能と消費エ
ネルギーの解析を行う.GPUでは,メモリとコアの動作周波数を変更可能である.本稿で
は,GPUがカーネルを実行する際には,動作周波数の変更を行わないものとする.本稿で
の解析は,最大の動作周波数で実行する場合と比較して消費エネルギーを削減可能な場
合がどういう場合であるかを調査する.まず,GPUで電圧と動作周波数を下げた場合に,
消費電力が下がるのかについて調査を行う.その後,CPUとGPU両方のワークロードを含
むベンチマークプログラムを用い消費エネルギーを削減可能な場合について考察してい
く.電圧と動作周波数を低下させた場合に,消費エネルギーが削減可能な場合を知るこ
とで,今後GPUを搭載したシステムにおいてDVFSを適用し,消費エネルギーを削減するア
ルゴリズムを考案するための知見を得る.したがって,GPUを搭載したシステムにおいて
は,CPU,GPUのコア,メモリのそれぞれで電圧と動作周波数を変更可能であるため,ど
ういった場合にどの電圧と動作周波数を下げるべきかについて解析する.
