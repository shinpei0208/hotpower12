\section{Related Work} 
\label{sec:related_work}

Nagasaka \textit{et al.} conjectured energy consumption of GPUs based on
the hardware performance counter~\cite{Nagasaka2010}.
This performance counter, however, is not adequate in that power
consumption rises even in an idle state when voltage and frequency are
scaled, though the performance counter does not change in an idle
state.
Hence, this approach would require an additional model to precisely
analyze the power consumption of the GPU.

Hong \textit{et al.} studied energy savings of GPUs, assuming power
gating available to limit the number of active
cores~\cite{Hong2009,Hong2010}.
In particular, they analyze PTX code to model the power and performance
of GPUs based on the number of instructions and memory accesses.
Unfortunately, none of the current GPU architectures yet supports power
gating, which limited their contribution to simulation studies.
Therefore, it is questionable if the presented power and performance
model is applicable to the real-world, and GPU power gating is also not
a realistic assumption at the moment.
Nonetheless, we consider that an offline PTX analysis for power and
performance prediction is a useful approach to the design of DVFS
algorithms.
What lacks in this approach, however, is a runtime analysis for input
data.
In this paper, we have analyzed the power and performance
characteristics depending on the size of input data. 

Lee \textit{et al.} presented a method to apply DVFS algorithms to the
GPU.
They particularly aimed at maximizing performance under the given power
constraint~\cite{Lee2011}.
A strong limitation of their work, however, is that the evaluation of
power consumption is based on a conceptual model but not on real-world
hardware.
They also failed to discuss how to determine the voltage and frequency.
In this paper, we have rather explored how to minimize the energy
consumption of GPU-accelerated systems using the cutting-edge real-world
hardware.

Jiao \textit{et al.} evaluated the power and performance of an old
NVIDIA GTX~280 GPU~\cite{Jiao2010}.
They examined compute-intensive and memory-intensive programs.
According to their analysis, energy consumption could often be reduced
by lowering the core clock when workload is memory-intensive.
This is exactly the same as what we have identified for an NVIDIA's
GTX~480 GPU.
Therefore, we conjecture that this observation and knowledge could be
applied to future GPU architectures as well.
In addition, we have disclosed that energy consumption could also be
reduced by scaling the memory clock.
This opens up a new insight into DVFS algorithms for GPU-accelerated
systems.

Ying \textit{et al.} analyzed the power and performance of an AMD
HD~5870 GPU using a random forest method with the profile
counter~\cite{ying2011}.
They revealed that activating a fewer number of ALUs reduces power
consumption.
However, this approach incurs an increase in execution time, and does
not successfully reduce energy consumption.
This is attributed to the fact that they use only software management.
Meanwhile, we have demonstrated that energy can be reduced by scaling
the voltate and frequency of the GPU.
