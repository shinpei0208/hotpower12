\section{Conclusion and Future Work}
\label{sec:conclusion}

We have presented a power and performance analysis of GPU-accelerated
systems based on the NVIDIA Fermi architecture.
Our findings include that the CPU is a weak factor for energy savings of
GPU-accelerated systems unless power gating is suppoted by the GPU.
In contrast, voltage and frequency scaling of the GPU is significant to
reduce energy consumption.
Our experimental results demonstrated that system energy could be reduced
by about 28\% retaining a decrease in performance within 1\%, if the
performance level of the GPU can be scaled effectively.

In future work, we plan to develop DVFS algorithms for GPU-accelerated
systems, using the characteristic identified in this paper.
We basically consider such an approach that controls the GPU core clock
for memory-intensive workload while controls the GPU memory clock for
compute-intensive workload.
To this end, we integrate PTX code analysis~\cite{Hong2009,Hong2010}
into DVFS algorithms so that energy optimization can be provided at
runtime.
We also consider a further dynamic scheme that scales the performance
level of the GPU during the execution of GPU code, whereas we restricted
a scaling point at the boundary of GPU code in this paper.