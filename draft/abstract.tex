\begin{abstract}

 Graphics processing units (GPUs) provide significant improvements in
 performance and performance-per-watt as compared to traditional
 multicore CPUs.
 This energy-efficiency of GPUs has facilitated use of GPUs in many
 application domains.
 Albeit energy efficient, GPUs still consume non-trivial power
 independently of CPUs.
 It is desired to analyze the power and performance charateristic of
 GPUs and their causal relation with CPUs.
 In this paper, we provide a power and performance analysis of
 GPU-accelerated systems for better understandings of these implications.
 Our analysis discloses that system energy could be
 reduced by about 28\% retaining a decrease in performance within 1\%.
 Specifically, we identify that energy saving is particularly significant
 when (i) reducing the GPU memory clock for compute-intensive workload
 and (ii) reducing the GPU core clock for memory-intensive workload.
 We also demonstrate that voltage and frequency scaling of CPUs is
 trivial and even should not be applied in GPU-accelerated systems.
 We believe that these findings are useful to develop dynamic voltage and
 frequency scaling (DVFS) algorithms for GPU-accelerated systems.

\end{abstract}
