\section{System Platform}
\label{sec:platform}

\begin{table}[!t]
 \caption{Performance levels of GTX 480 (GPU).}
 \vspace{-2.0mm}
 \label{tab:GPU-frequency}
 \footnotesize
 \begin{center}
  \begin{tabular}{|c|c|c|c|}
   \hline
   Clock Domains &Min [MHz]&Low [MHz] & High [MHz]\\
   \hline
   \hline
    Core & 50 & 405 & 700\\
    \hline
    Memory & 135 & 324 & 1848\\
   \hline
  \end{tabular}
  \vspace{-5.0mm}
 \end{center}
\end{table}

\begin{table}[!t]
 \caption{Performance levels of Core i5 2400 (CPU).}
 \small
 \label{tab:CPU-frequency}
 \begin{center}
  \begin{tabular}{|c|c|c|c|}
   \hline
   Platforms&Min [MHz]&Low [MHz]&High [MHz]\\
   \hline
   \hline
   Core i5-2400 &1600& 2700 & 3300.1\\
   \hline
  \end{tabular}
 \end{center}
 \end{table}

We use an NVIDIA GeForce GTX 480 graphics card and Intel Core i5 2400
processor with the Linux kernel 3.3.0.
Table~\ref{tab:GPU-frequency}~and~\ref{tab:CPU-frequency} present their
available performance levels respectively.
To perform the experiment, we use NVIDIA's proprietary
software~\cite{BLOB} and Gdev~\cite{Kato2012} case by case.
NVIDIA's proprietary software does not provide a system interface to
scale the performance level of the GPU.
We hence provide the modified BIOS files for the GPU that force the
binary driver to configure the GPU with the specified performance level
when loaded.
This method enables us to choose any set of the GPU core and memory
clocks, but requires the driver to reload, and the configuration is
static while the driver is running.
On the other hand, Gdev allows the system to change the performance
level of the GPU dynamically at runtime through the Linux ``/proc'' file
system interface.
However, it is available only for the GPU core clock at the moment, and
the GPU memory clock is fixed at 135MHz.
This is limited due to an open-source implementation of Linux, but will
be removed in the future release.

The experimental workload executes the Rodinia benchmark suite
2.0.1~\cite{Che2010} and our original microbenchmark programs of matrix
computation.
All input data for the Rodinia programs use the maximum feasible size,
while the microbenchmark programs vary data size to conduct fine-grained
measurements, all of which are written in CUDA.
We use the NVIDIA CUDA Compiler (NVCC) 4.0~\cite{CUDA40} to compile the
programs.
Both NVIDIA's proprietary software and Gdev receive the same program
binary.

The power and energy consumption of the system is measured by the
YOKOGAWA WT1600 digital power meter.
This instrument obtains the voltage and electric current every $50ms$
from the power plug of the machine.
The power consumption is calculated by multiplying the voltage and
current, while the energy consumption is derived by accumulation of
power consumption.
